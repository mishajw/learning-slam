\documentclass{article}

\usepackage{biblatex}
\addbibresource{main.bib}

\title{Learning SLAM}
\author{Misha Wagner}

\begin{document}

\maketitle
\abstract{High-level notes on how SLAM works, with an accompanying Rust
implementation. Compiled from Cyrill Stachniss's course\cite{stachnisscourse}
and personal research}

\section{What is SLAM?}

Simultaneous Localisation and Mapping, as it says on the tin, is a task where
the goal is to simultaneously predict:
\begin{itemize}
  \item A map of the environment
  \item An agent's location in the map
\end{itemize}
\dots given, at each timestep:
\begin{itemize}
  \item What the agent can see
  \item How the agent is moving.
\end{itemize}

In literature, the predicted map and location (the state) is denoted as $x$,
what is seen is denoted as $z$, and how we move is denoted as $u$. Therefore,
the SLAM problem is to find a way to calculate this function:
\begin{equation}
  p(x | z, u)
\end{equation}
\dots for any given $x, z, u$, and find the $x$ that maximises this function.

\subsection{Representations}

How do we represent each of the map, location, what's seen, and how the agent
is moving?

TODO

\subsection{Computation outline}

\section{Feature detection}

\section{SLAM algorithms}

\subsection{Bayes filter}

\subsection{Kalman filter}

\subsection{Extended Kalman filter}

\subsection{Unscented Kalman filter}

\subsection{Extended information filter}

\subsection{Sparse extended information filter}

\printbibliography{}

\end{document}
